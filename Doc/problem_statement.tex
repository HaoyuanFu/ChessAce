\documentclass[12pt,letterpaper]{article}
\usepackage[utf8]{inputenc}
\usepackage{amsmath}
\usepackage{amsfonts}
\usepackage{amssymb}
\usepackage[left=1in, right=1in, top=0.5in, bottom=1in]{geometry}
\title{Chess: Problem Statement}
\author{Group 18: Jerry Ke, Mengshan Cui, Harry Fu}
\date{}

\begin{document}
\maketitle
\section{Introduction}
Large competitive games are popular nowadays, but there are still many people do not have much free time can spend on that, or some people just would like to play some portable, offline board game like chess or checkers which they could spend little free time on it and show their skill. For example, if two board game fans want to play board game during their study break, they don’t have time and space to set the board and all the components, in this case, an offline, portable, player versus player board game is needed to allow users to play it face to face anywhere anytime. That is the reason we want to reimplement an offline java-based pc chess game to let them play chess on any personal computer or laptop conveniently. 
\section{Importance}
Nowadays, existing online chess platforms indubitably provide chances of a fair game between players with the same rank worldwide. However, a platform demands relatively excess operations, memory space and more importantly, internet access for an occasional casual game between friends and family members. 
Unfortunately, if such users go seeking offline chess at this point, they will find out most of the chess programs are either paid or lack of PVP ability. A few of them even forget to implement several significant unit movements, such as castling, from the rule of chess. 

\section{Context}
The re-implement of this software is a game that can be played by all ages. It can be lauched on any pc platform, whether using a Mac, Windows, or Linux operating system environment. ChessOOP creates an environment for users to play the game for free and without any downloads allowing for a more accessible way to play this game. However, this software lacks some of formal documentations and part of the game rule is not achieved, such as castling. The stakeholders for this software are developers, clients, consumers and future developers. MIF team is the developers and clients for this project, and consumers will be the end-users who will play the game after the launch. The scope of this project will be to re-implement the program as a new program with appropriate documentation. This will make it easier for users to operate and for developers to contribute to the open-source software in the future.

\pagenumbering{gobble}
\end{document}
