\documentclass[12pt,letterpaper]{article}
\usepackage[utf8]{inputenc}
\usepackage{amsmath}
\usepackage{amsfonts}
\usepackage{amssymb}
\usepackage[left=1in, right=1in, top=0.5in, bottom=1in]{geometry}
\title{Chess: Problem Statement}
\author{Group 18: Jerry Ke, Mengshan Cui, Harry Fu}
\date{}

\begin{document}
\maketitle
\section{Introduction}

\section{Importance}
Currently, there exist multiple online chess platforms so that people can quickly queue into a fair game with other players. However, a platform requires relatively excess operations, and more importantly, internet access for an occasional offline game between friends and family members. On the other hand, the available offline chess games are either paid, lack of PVP ability, or miss significant unit movements from the rule of chess. 

\section{Context}
The re-implement of this software is a game that can be played by all ages. It can be lauched on any pc platform, whether using a Mac, Windows, or Linux operating system environment. ChessOOP creates an environment for users to play the game for free and without any downloads allowing for a more accessible way to play this game. However, this software lacks some of formal documentations and part of the game rule is not achieved, such as castling. The stakeholders for this software are developers, clients, consumers and future developers. MIF team is the developers and clients for this project, and consumers will be the end-users who will play the game after the launch. The scope of this project will be to re-implement the program as a new program with appropriate documentation. This will make it easier for users to operate and for developers to contribute to the open-source software in the future.

\pagenumbering{gobble}
\end{document}
