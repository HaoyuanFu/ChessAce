\documentclass[12pt,letterpaper]{article}
\usepackage[utf8]{inputenc}
\usepackage{amsmath}
\usepackage{amsfonts}
\usepackage{amssymb}
\usepackage[left=1in, right=1in, top=0.5in, bottom=1in]{geometry}
\title{ChessAce: Problem Statement}
\author{Group 18: Jerry Ke, Mengshan Cui, Harry Fu}
\date{}

\begin{document}
\maketitle
\section{Introduction}
Large-scale video games are popular at present, but during the business days, the majority of the users, including students and employees, cannot enjoy such games with their insufficient fragmented time. Fortunately, people still have traditional board games like chess to utilize such spare time. Furthermore, chess can also help establish and enhance the relationship between different people. For example, two board game fans meet and choose to play chess during their study break. However, setting up the physical board and pieces usually depletes the limited time and the interest that appears occasionally. In this case, a portable, player versus player(PVP) chess program is needed to allow users to play it face to face anywhere anytime. 

\section{Importance}
Nowadays, existing online chess platforms indubitably provide chances of a fair game between players with the same rank worldwide. However, an online platform demands relatively excess operations, memory space and more importantly, internet access for an occasional casual game between friends and family members. 
\medskip
\newline
Unfortunately, if such users go seeking offline chess at this point, they will find out most of the chess programs are either paid, lack of PVP ability or unable to execute on different operating systems. A few of them even forget to implement several significant unit movements, such as castling, from the rule of chess. 
Therefore, re-implementing an offline, multi-platform, rule-complete chess game is necessary so that players can enjoy chess on any personal computers or laptops conveniently. 

\section{Context}
This software is a chess game that can be played by all ages. It can be launched on any pc platform, whether using a Mac, Windows, or Linux operating system environment. The stakeholders for this software are developers, users, and future developers. Members of the MIF team are the developers for this project, and users are individuals who will play the game after the launch. 
\medskip
\newline
ChessOOP is an offline open-source software which creates an environment for users to play the game for free and allows a more accessible way. However, this software lacks formal documentation and does not achieve part of the game rule. The scope of this project will be to re-implement the program as a new program called ChessAce with appropriate documentation and improvement on game rule. This will make it easier for users to operate and for developers to contribute to the open-source software in the future.

\pagenumbering{gobble}
\end{document}
